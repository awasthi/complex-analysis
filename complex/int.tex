%\chapter{Complex Integration}\index{Complex integration}
\section{Integration in complex plain}
In case of real variable, the path of the integration of $\ds \int_a^b f(x)dx$ is always along the $x$-axis from $x=a$ to $x=b$. But in case of a complex function $f(z)$ the path of a complex function $f(z)$ the path of the definite integral $\ds \int_{\alpha}^{\beta} f(z)dz$ can be along any curve from $z=\alpha$ to $z=\beta$. 
\begin{example}
Evaluate $\ds \int_{0}^{2+i} \bar{z}^2 dz$ along the real axis from $z=0$ to $z=2$ and then along parallel to $y$-axis from $z=2$ to $z=2+i$.
\end{example}
\begin{solution}
\begin{align*}
	\int_{0}^{2+i} \bar{z}^2 dz		&= \int_{0}^{2+i} (x-iy)^2 (dx+idy)\\
																&= \int_{0}^{2+i} (x^2-y^2-2ixy) (dx+idy)\\
\end{align*}
\begin{figure}[ht]
  \begin{center}
\scalebox{0.7} % Change this value to rescale the drawing.
{
\begin{pspicture}(0,-1.08)(2.9934375,1.05)
\psline[linewidth=0.02cm,arrowsize=0.05291667cm 2.0,arrowlength=1.4,arrowinset=0.4]{<-}(0.2778125,1.04)(0.2778125,-0.74)
\psline[linewidth=0.02cm,arrowsize=0.05291667cm 2.0,arrowlength=1.4,arrowinset=0.4]{->}(0.2778125,-0.74)(2.7178125,-0.74)
\psline[linewidth=0.02cm,arrowsize=0.05291667cm 2.0,arrowlength=1.4,arrowinset=0.4]{->}(0.2778125,-0.74)(1.6978126,0.56)
\psline[linewidth=0.02cm,arrowsize=0.05291667cm 2.0,arrowlength=1.4,arrowinset=0.4]{<-}(1.6778125,0.58)(1.6778125,-0.74)
\usefont{T1}{ptm}{m}{n}
\rput(1.7167188,-0.93){z=2}
\usefont{T1}{ptm}{m}{n}
\rput(2.2298439,0.55){z=2+i}
\usefont{T1}{ptm}{m}{n}
\rput(2.8632812,-0.73){x}
\usefont{T1}{ptm}{m}{n}
\rput(0.085625,0.93){y}
\end{pspicture} 
}
\end{center}
\caption{}
\end{figure}
\paragraph{Along real axis from $z=0$ to $z=2$ (y=0)}:
\[y=0 \Rightarrow dy=0, dz = d(x+iy) = dx \]
\[z=0, y=0 \Rightarrow x=0\]
and
\[z=2, y=0 \Rightarrow x=2\]
\begin{align*}
	\int_{0}^{2+i} \bar{z}^2 dz		&= \int_{0}^{2} (x^2) (dx)\\
	&= \left[\frac{x^3}{3} \right]_0^2 = \frac{8}{3}\\
\end{align*}
\paragraph{Along parallel to $y$-axis from $z=2$ to $z=2+i$ (x=2)}
\[x=2 \Rightarrow dx=0, dz = d(x+iy) = idy\]
\[z=2, x=2 \Rightarrow y=0~~~~~and~~~~~z=2+i, x=2 \Rightarrow y=1\]
\begin{align*}
	\int_{0}^{2+i} \bar{z}^2 dz		&= \int_{0}^{1} (4-y^2-4iy) (i.dy)\\
	&= i \left[4y - \frac{y^3}{3} - 4i\frac{y^2}{2} \right]_0^1 =\left[\frac{11}{3}i + 2\right]\\
\end{align*}
$\ds \int_{0}^{2+i} \bar{z}^2 dz$ along the real axis from $z=0$ to $z=2$ then along parallel to $y$-axis from $z=2$ to $z=2+i$
\[= \frac{8}{3} +  \frac{11}{3}i + 2  = \frac{1}{3}(14+11i)\]
\end{solution}
\begin{problems}
\prob Find the value of the integral 
\[\int_0^{1+i}(x-y+ix^2)dz\]
\subprob Along the straight line from $z=0$ to $z=1+i$.
\begin{sol}\[\frac{1}{3}(i-1)\]\end{sol}
\subprob along the real axis from $z=0$ to $z=1$ and then along parallel to $y$-axis from $z=1$ to $z=1+i$.
\begin{sol}\[-\frac{1}{2}+\frac{5}{6}i\]\end{sol}
\prob Integrate $f(z) = x^2 + ixy$ from $A(1,1)$ to $B(2,8)$ along
\subprob the straight line $AB$
\begin{sol}
Equation of line $AB$
\[y=7x-6\]
Now 
\[\int f(z) dz  = \int (x^2 + ixy)(dx+idy)\]
substitute $y=7x-6$, $dy=7dx$, then on integrating
\[=\frac{1}{3}[-147+71i]\]
\end{sol}
\subprob the curve $C$, $x=t, ~y=t^3$.
\begin{sol}
Here
\[\int f(z) dz  = \int (x^2 + ixy)(dx+idy)\]
substitute $x=t,y=t^3$, $dx=dt\text{ and } dy=3t^2dt$, then on integrating with respect to $t$ from 1 to 2
\[=-\frac{1094}{21}+\frac{124}{5}i\]
\end{sol}

\prob Evaluate the integral $\ds \int_c (3y^2dx+2ydy)$, where $c$ is the circle $x^2+y^2=1$, counterclockwise from $(1,0)$ to $(0,1)$.
\begin{sol}
-1
\end{sol}
\end{problems}

\section{Cauchy's Integral Theorem}\index{Cauchy's Integral Theorem}
\begin{theorem}
If a function $f(z)$ is analytic and its derivative $f'(z)$ continuous at all points within and on a simple closed curve $c$, then $\ds \int_c f(z) dz = 0$.
\end{theorem}
\begin{proof}
Let $f(z)=u+iv$ and $z=x+iy$ and region enclosed by the curve $c$ be $R$, then
\begin{align*}
	\int_cf(z)dz 	&= \int_c(u+iv)(dx+idy) =  \int_c(udx-vdy) + i(vdx+udy)\\
								&= \int\int_R (-v_x - u_y)dxdy + i\int\int_R (u_x - v_y)dxdy\\
								\text{By Cauchy-Riemann equations,} \\
								&= \int\int_R (u_y - u_y)dxdy + i\int\int_R (u_x - u_x)dxdy =0
\end{align*}
\end{proof}
\begin{example}
Find the integral $\ds \int_c \frac{3z^2+7z+1}{z+1}dz$, where $c$ is the circle $\ds |z|=\frac{1}{2}$.
\end{example}
\begin{solution}
Poles of integrand are given by
$$z+1 = 0$$
That is, $z=-1$. Since given circle $|z|=\frac{1}{2}$, with centre $z=0$ and radius $1/2$ does not enclose $z=-1$. Thus it is obvious that the integrand is analytic everywhere. Hence, by Cauchy's Theorem,
\[\int_c \frac{3z^2+7z+1}{z+1}dz = 0\]
\end{solution}
\begin{theorem}[Cauchy's integral theorem for multi-connected region]\index{Cauchy's Integral Theorem!multi-connected region}
If a function $f(z)$ is analytic in region $R$ between two simple closed curves $c_1$ and $c_2$, then
\[\int_{c_1} f(z) dz = \int_{c_2} f(z) dz\]
\end{theorem}
\begin{proof}
Since $f(z)$ is analytic in region $R$, By Cauchy's Theorem
\[\int f(z) dz = 0\]
where path of integration is along $AB$, and curves $C_2$ in clockwise direction and along $BA$ and along $C_1$ in anticlockwise direction.

We may write,
\[\int_{AB} f(z) dz - \int_{c_2} f(z) dz + \int_{BA} f(z) dz + \int_{c_1} f(z) dz = 0\]
or
\[- \int_{c_2} f(z) dz +  \int_{c_1} f(z) dz = 0\]
\[ \int_{c_1} f(z) dz =  \int_{c_2} f(z) dz \]
\end{proof}
\section{Cauchy Integral Formula}\index{Cauchy Integral Formula}
\begin{theorem}
If a function $f(z)$ is analytic within and on a closed curve $c$, and if $a$ is any point within $c$ , then
\[f(a) = \frac{1}{2\pi i}\int_{z} \frac{f(z)}{(z-a)} dz\]
\end{theorem}
\begin{proof}
Let $z=a$ be a point within a closed curve $c$. Describe a circle $\gamma$ such that $|z-a|=\rho$ and it lies entirely within $c$. Now consider the function 
\begin{figure}[ht]
	\centering
\scalebox{0.7} % Change this value to rescale the drawing.
{
\begin{pspicture}(0,-1.4300019)(3.361875,1.4300019)
\definecolor{color73b}{rgb}{0.8,0.8,1.0}
\pswedge[linewidth=0.04,fillcolor=color73b](1.4100019,0.0){1.4100019}{275.4772}{272.41934}
\pscircle[linewidth=0.032,dimen=outer,fillstyle=solid](1.4510782,-0.023076477){0.9110782}
\psline[linewidth=0.04cm,arrowsize=0.05291667cm 2.0,arrowlength=1.4,arrowinset=0.4]{->}(0.0,0.009998169)(0.0,-0.15000182)
\psline[linewidth=0.04cm,arrowsize=0.05291667cm 2.0,arrowlength=1.4,arrowinset=0.4]{->}(0.54,-0.15000182)(0.56,0.029998168)
\psline[linewidth=0.04cm,arrowsize=0.05291667cm 2.0,arrowlength=1.4,arrowinset=0.4]{->}(2.32,0.16999817)(2.32,-0.11000183)
\psline[linewidth=0.04cm,arrowsize=0.05291667cm 2.0,arrowlength=1.4,arrowinset=0.4]{->}(2.82,0.16999817)(2.84,-0.03000183)
\usefont{T1}{ptm}{m}{n}
\rput(3.0214062,-0.68000185){$C$}
\usefont{T1}{ptm}{m}{n}
\rput(2.1914062,0.5999982){$\gamma$}
\psdots[dotsize=0.12](1.42,-0.03000183)
\usefont{T1}{ptm}{m}{n}
\rput(1.4614062,-0.16000183){$a$}
\end{pspicture} 
}
\caption{Cauchy Integral Formula}
\end{figure}
\[\phi(z) = \frac{f(z)}{(z-a)}\]
Obviously, this function is analytic in region between $\gamma$ and $c$. Hence by Cauchy's integral theorem for multiconnected region, we have
\[\int_{c} \phi(z) dz = \int_{\gamma} \phi(z) dz\]
or
\begin{align*}
	\int_{c} \frac{f(z)}{(z-a)} dz 	&= \int_{\gamma} \frac{f(z)}{(z-a)} dz \\
																	&= \int_{\gamma} \frac{f(z) - f(a) + f(a)}{(z-a)} dz \\
																	&= \int_{\gamma} \frac{f(z) - f(a)}{(z-a)} dz + \int_{\gamma} \frac{f(a)}{(z-a)} dz \\
																	&= I_1 + I_2
\end{align*}
Now, since $|z-a|=\rho$, we have $z=a+ \rho e^{i\theta}$ and $dz = i \rho e^{i\theta} d\theta$. Hence
\begin{align*}
 I_1 &= \int_{\gamma} \frac{f(z) - f(a)}{(z-a)} dz \\
 	 &= \int_{0}^{2\pi} \frac{f(a+ \rho e^{i\theta}) - f(a)}{[(a+ \rho e^{i\theta})-a]} i \rho e^{i\theta} d\theta \\
 	 &= \int_{0}^{2\pi} [f(a+ \rho e^{i\theta}) - f(a)] i d\theta \\
 	 &=0 ~~~~~~~~\text{ as } \rho \text{ tends to 0}
\end{align*}
and
\begin{align*}
 I_2 &= \int_{\gamma} \frac{f(a)}{(z-a)} dz \\
 	 &= \int_{0}^{2\pi} \frac{f(a)}{[(a+ \rho e^{i\theta})-a]} i \rho e^{i\theta} d\theta \\
 	 &= f(a) \int_{0}^{2\pi} i d\theta \\
 	 &= 2\pi i f(a)	 
 \end{align*}
Hence,
\[\int_{c} \frac{f(z)}{(z-a)} dz = I_1 + I_2\]
That is 
\[\int_{c} \frac{f(z)}{(z-a)} dz = 0 + 2 \pi i f(a) \]
or
\[ f(a) = \frac{1}{2 \pi i}\int_{c} \frac{f(z)}{(z-a)} dz \]
\end{proof}
\begin{example}
Evaluate (i) $\ds \int_c \frac{e^z}{z+2}dz$ and (ii) $\ds \int_c \frac{e^z}{z}dz$, where $c$ is circle $|z|=1$.
\end{example}
\begin{solution}
(i) The function $\ds \frac{e^z}{z+2}$ is analytic everywhere except at $z=-2$. This point lies outside the circle $|z|=1$. Thus function is analytic within and on $c$, by Cauchy's Theorem, we have
\[\int_{|z|=1} \frac{e^z}{z+2}dz = 0\]

(ii)
The function $\ds \frac{e^z}{z}$ is analytic everywhere except at $z=0$. The point $z=0$ strictly inside $|z|=1$. Hence by Cauchy's Integral formula, we have
\[\int_{|z|=1}~ \frac{e^z}{z}dz = 2 \pi i (e^z)_{z=0} =  2 \pi i \]
\end{solution}
\section{Cauchy Integral Formula For The Derivatives of An Analytic Function}
\begin{theorem}
If a function $f(z)$ is analytic within and on a closed curve $c$, and if $a$ is any point within $c$ , then its derivative is also analytic within and on closed curve $c$, and is given as 
\[f'(a) = \frac{1}{2\pi i}\int_{c} \frac{f(z)}{(z-a)^2} dz\]
\end{theorem}
\begin{proof}
We know Cauchy's Integral formula
\begin{align*}
	f(a) &= \frac{1}{2\pi}\int_{c} \frac{f(z)}{(z-a)} dz \\
\text{Differentiating, wrt $a$ , we get} \\
	f'(a) &= \frac{1}{2\pi}\frac{d}{da}\left[\int_{c} \frac{f(z)}{(z-a)} dz \right]\\
	      &= \frac{1}{2\pi} \int_{c} f(z) \frac{\partial}{\partial a}\left[\frac{1}{(z-a)} \right]dz\\
	      &= \frac{1}{2\pi}\int_{c} \frac{f(z)}{(z-a)^2} dz
\end{align*}
We may generalize it,
\[f^n(a) = \frac{n!}{2\pi i}\int_{c} \frac{f(z)}{(z-a)^{n+1}} dz\]
\end{proof}
\begin{example}
Evaluate the following integral $\ds \int_{c}\frac{1}{z}\cos z dz$, where $c$ is the ellipse $9x^{2}+4y^{2}=1$.
\end{example} 
\begin{solution}
Here function $\ds \frac{1}{z}\cos z$ has a simple pole at $z=0$. The given ellipse $9x^{2}+4y^{2}=1.$ encloses pole $z=0$.

\noindent
By Cauchy Integral formula
\[\int_{c}\frac{cosz}{z}dz=2\pi i(cosz)_{z=0}=2\pi i\]
\end{solution} 
\begin{example}
Evaluate the complex integral $\ds \int_{c} \tan z dz$, where $c$ is $|z|=2$.
\end{example} 
\begin{solution}
We have
\[\int_{c}tanz.dz=\int_{c}\frac{\sin z}{\cos z} dz\]
$|z|=2$, is a circle with centre at origin and radius = 2. Poles are given by putting the denominator equal to zero. i.e.,
\[\cos z=0 ~~~~\Rightarrow ~~~~ z=-\frac{\pi}{2},\frac{\pi}{2},\frac{3\pi}{2},...\]
The integrand has two poles at $z=\frac{\pi}{2}$ and $z=-\frac{\pi}{2}$ inside the given circle $|z|=2$.

\noindent
On applying Cauchy integral formula 
\begin{align*}
	\int_{c}\frac{\sin z}{\cos z}dz &= \int_{c_1}\frac{\sin z}{\cos z}dz + \int_{c_2} \frac{\sin z}{\cos z}dz \\
	&=2\pi i[\sin z]_{z=\frac{\pi}{2}} + 2 \pi i[\sin z]_{z=-\frac{\pi}{2}} \\
	&=2\pi i(1)+2\pi i(-1)=0
\end{align*}
\end{solution} 
\begin{example}
Evaluate  $\ds \int_{c}\frac{e^{z}}{z^{2}+1}dz$ over the circular path $|z|=2$.
\end{example} 
\begin{solution}
Here,
\[z^{2}+1=0,~~z^{2}=-1,~~z=\pm i\]
\begin{figure}[ht]
  \begin{center}
   \scalebox{0.5} % Change this value to rescale the drawing.
{
\begin{pspicture}(0,-3.0)(6.0,3.0)
\rput(3.0,0.0){\psaxes[linewidth=0.02,labels=none,ticks=y,ticksize=0.10583333cm,showorigin=false](0,0)(-3,-3)(3,3)}
\pscircle[linewidth=0.02,dimen=outer](2.94,-0.02){2.0}
\pscircle[linewidth=0.02,dimen=outer](2.94,0.98){0.3}
\pscircle[linewidth=0.02,dimen=outer](2.94,-1.02){0.3}
\usefont{T1}{ptm}{m}{n}
\rput(3.2803125,1.13){i}
\usefont{T1}{ptm}{m}{n}
\rput(3.4182813,-0.93){-i}
\end{pspicture} }
  \end{center}
  \caption{}
\end{figure}

Both points are inside the given circle with at origin and radius 2.
\begin{align*}
\int_{c}\frac{1}{2i}\{\frac{e^{z}}{z-i}-\frac{e^{z}}{z+i}\}dz &=\int_{c}\frac{1}{2i}\frac{e^{z}}{z-i}dz-\frac{1}{2i}\int_{c}\frac{e}{z+i}dz\\
&=\frac{1}{2i}[2\pi i(e^{z})_{z=i}-2\pi i(e^{z})_{z=-i}] \\
&=\frac{2\pi i}{2i}[e^{i}-e^{-i}]=2\pi i \sin(1) \\
\end{align*}
\paragraph{Second Method} 
\begin{align*}
	\int_{c}\frac{e^{z}}{z^{2}+1}dz &=\int_{c}\frac{e^{z}d_{z}}{(z+i)(z-i)} \\
	&=\int_{c1}\frac{\frac{e^{z}}{z-i}}{z+i}dz+\int_{c}\frac{\frac{e^{z}}{z+i}}{z-i}dz \\
	&=2\pi i(\frac{e^{z}}{z-i})_{z=-i}+2\pi i(\frac{e^{z}}{z+i})_{z=i} \\
	&=[2\pi i\frac{e^{-i}}{-i-i}+2\pi i\frac{e^{i}}{i+i}]=\pi[-e^{-i}+e^{i}] \\
	&=\pi(2isin1)=2\pi i \sin 1
\end{align*}
\end{solution} 
\begin{example}
Evaluate $\ds \int_{c}\frac{z-1}{(z+1)^{2}(z-2)}dz$ where $c$ is $|z-i|= 2$.
\end{example} 
\begin{solution}
The centre of the circle is at $z = i$ and its radius is $2$. Poles are obtained by putting the denominator equal to zero.
\[(z+1)^{2}(z-2)=0~~~~\Rightarrow ~~~~z=-1,-1,2\]
The integral has two Poles at $z = -1$ (second order) and $z = 2$ (simple pole ) of which $z = -1$ is inside the given circle.
We can rewrite
\[\int_{c}\frac{(z-1)dz}{(z+1)^{2}(z-2)}=\int_{c1}\frac{\frac{z-1}{z-2}}{(z+1)^{2}}dz\]
By Cauchy Integral formula 
\[\ds \int\frac{f(z)}{(z+1)^{2}}dz=2\pi i f'(-1)\]
Here
\begin{align*}
f(z)&=\frac{z-1}{z-2} \\
f'(z)&=\frac{(z-2).1-(z-1).1}{(z-2)^{2}}=\frac{-1}{(z-2)^{2}}=\frac{-1}{(z-2)^{2}}\\
\Rightarrow ~~~f'(-1)&=\frac{-1}{(-1-2)^{2}}=\frac{-1}{9} \\
\therefore \int \frac{(z-1)}{(z+1)^{2}(z-2)}dz &= =-\frac{2\pi i}{9}
\end{align*}
\end{solution} 
\begin{example}
Use Cauchy integral formula to evaluate .
\[\int_{c}\frac{sin\pi z^{2}+cos\pi z^{2}}{(z-1)(z-2)}dz\]
where $c$ is the circle $|z| = 3$.
\end{example} 
\begin{solution}
Poles of the integrand are given by putting the denominator equal
to zero.
\[(z-1)(z-2)=0, ~~~~z=1,2\]
The integrand has two poles at $z = 1,2$. The given circle $|z| = 3$ with centre at $z = 0$ and radius 3 encloses both the poles $z = 1$, and $z = 2$.
\begin{align*}
	\int_{c}\frac{sin\pi z^{2}+cos\pi z^{2}}{(z-1)(z-2)}dz &=\int_{c_1}{\frac{\frac{sin\pi z^{2}+cos\pi z^{2}}{(z-2)}}{(z-1)}dz}+\int_{c_2}{\frac{\frac{sin\pi z^{2}+cos\pi z^{2}}{(z-1)}}{(z-2)}dx} \\
	&=2\pi i \left[\frac{sin\pi z^{2}+cos\pi z^{2}}{z-2}\right]_{z=1}+2\pi i\left[\frac{sin\pi z^{2}+cos\pi z^{2}}{z-1}\right]_{z=2} \\
	&=2\pi i \left[\frac{sin\pi+cos\pi}{1-2}\right]+2\pi i\left[\frac{sin4\pi+cos4\pi}{2-1}\right] \\
	&=2\pi i\left(\frac{-1}{-1}\right)+2\pi i\left(\frac{1}{1}\right)=4\pi i 
\end{align*}
\end{solution}
\begin{problems}
\prob Evaluate the following by using Cauchy integral formula:

\subprob $\ds \int_{c}\frac{1}{z-a}dz,$ where $c$ is a simple closed curve and
the point $\ds z = a$ is  (i) outside $c$; (ii) inside $c$. 
\begin{sol}
(i) 0	~~~~~~~~ (ii) $2\pi i$
\end{sol}
\subprob  $\ds \int_{c}\frac{e^{z}}{z-1}dz$, where c is the circle $|z| =2$.
\begin{sol}
$2\pi i e$
\end{sol}
\subprob $\ds \int_{c}\frac{cos\pi z}{z-1}dz$, where c is the circle $|z|=3$.
\begin{sol}
$-2\pi i$
\end{sol}
\subprob $\ds \int_{c}\frac{cos\pi z^{2}}{(z-1)(z-2)}dz$, where c is the circle $|z| = 3$. 
\begin{sol}
$4\pi i $
\end{sol}
\subprob $\ds \int_{c}\frac{e^{-z}}{(z+2)^{5}}dz$, where c is the circle $|z|=3$. 
\begin{sol}
$\frac{i\pi e^2}{12}$
\end{sol}
\subprob $\ds \int_{c}\frac{e^{2z}}{(z+1)^{4}}dz$, where c is the circle $|z|= 2$. 
\begin{sol}
$\frac{8 i \pi e^{-2}}{3}$
\end{sol}
\subprob $\ds \int_{c}\frac{3z^{2}+z}{z^{2}-1}dz$, where c is the circle $|z-1|=1$.
\begin{sol}
$4 \pi i$
\end{sol}
\prob Evaluate the following integral using Cauchy integral formula 
\[\int_{c}\frac{4-3z}{z(z-1)(z-2)}dz\]
where $c$ is the circle $|z| =\frac{3}{2}$.
\begin{sol}
$-12 \pi i$
\end{sol}
\prob Integrate $\ds  \frac{1}{(z^{3}-1)^{2}}$ the counter clockwise sense around the circle $|z-1|=1$.
\begin{sol}
$-\frac{4i\pi}{9}$
\end{sol}
\prob Find the value of $\ds \int_{c}\frac{2z^{2}+z}{z^{2}-1}dz$, if $c$ is circle of unit radius with centre at $z = 1$.
\begin{sol}
$3\pi i$
\end{sol}

\end{problems}