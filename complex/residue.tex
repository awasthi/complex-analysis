%\chapter{Residue: Evaluation of Real Integrals}\index{Evaluation of Real Integrals}\index{Residue}
\section{Residue Theorem}\index{Residue Theorem}
\subsection{Residue}\index{Residue}
 By Laurent series expansion of an analytic function $f(z)$, we have
 \[f(x) = \sum_{n=-\infty}^{\infty}  a_n(z-a)^n\]
 where \[a_n = \frac{1}{2\pi i} \int_C \frac{f(t)}{(t-a)^{n+1}} dt\]
As we have from definition, the coefficient of $\frac{1}{(z-a)}$ in Laurent expansion about $z=a$ is called as \emph{Residue}, Which may be obtained by putting $n=-1$ in above equation. Thus,
\[RES(a) =  \frac{1}{2\pi i} \int_C f(t) dt\]
which implies
\[\int_C f(t) dt = 2 \pi i .RES(a)\]
\subsection{Residue Theorem}\index{Residue Theorem}
\begin{theorem}
If $f(z)$ be analytic within and on a simple closed curve $C$ except at number of poles. (Say $z_1, z_2, z_3, ..., z_n$ are poles.) then the integral
\begin{center}
$\ds \int_C f(t) dt = 2 \pi i$(Sum of Residue of $f(z)$ at each poles)
\end{center}
\end{theorem}
\begin{proof}
Let $f(z)$ be analytic within and on simple closed curve $C$ except at number of poles $z_1, z_2, z_3, ..., z_n$.

Let $C_1, C_2, ..., C_n$ be small circles with centres at $z_1, z_2, z_3, ..., z_n$ respectively. Then by Cauchy extension theorem, we have
\[\int_C f(z) dz = \int_{C_1} f(z) dz + \int_{C_2} f(z) dz + ... + \int_{C_n} f(z) dz \] 
Now, we have,
\[\int_{C_1} f(z) dt = 2 \pi i .RES(z_1) \]
\[\int_{C_2} f(z) dt = 2 \pi i .RES(z_2) \]
\[... ~~...~~...\]
\[\int_{C_n} f(z) dt = 2 \pi i .RES(z_n)\]
Therefore, 
\[\int_C f(z) dz = 2 \pi i .RES(z_1) +2 \pi i .RES(z_2) +...+2 \pi i .RES(z_n)  \] 
\[\int_C f(z) dz = 2 \pi i .[RES(z_1) +RES(z_2) +...+RES(z_n)]  \] 
\[\int_C f(z) dz = 2 \pi i .[\textrm{Sum of all Residues at each poles.}]  \] 
\[\int_C f(z) dz = 2 \pi i .Res  \] 
We use $Res$ for `Sum of Residues'\index{Sum of Residue} through out the text.  
\end{proof}
\section{Evaluation of Real Integrals by Contour Integration}\index{Contour integration}\index{Real Integrals}
\subsection*{Type $\int_0^{2 \pi} f(\cos \theta, \sin \theta) d\theta$}
For such questions consider a unit radius circle with center at origin, as contour
\[|z| = 1~~~~~~\Rightarrow    z=e^{i\theta}~~~~~~~~~~~~d\theta ={dz \over iz}\]
Now use following relations
\[\cos \theta = {e^{i\theta} + e^{-i\theta} \over 2}  = {1 \over 2}(z + {1 \over z}) ~~~~~~~\]
\[\sin \theta = {e^{i\theta} - e^{-i\theta} \over 2i}={1 \over 2i}(z - {1 \over z})\] 
We get, the whole function is converted into a function of $f(z)$, Now integral become 
\[I = \int_C f(z) dz\]
where C is unit circle. The value of this integral may be obtained by using Residue Theorem, which is $2\pi i.$(Sum of Residue inside $C$)
\subsection{Form I}
Form $\int_0^{2\pi} {1 \over a+b \cos\theta} d\theta$ or $\int_0^{2\pi} {1 \over a+b \sin\theta} d\theta$
\begin{example}
Use residue calculus to evaluate the following integral$\int_{0}^{2\pi}\frac{1}{5-4sin\theta}d\theta$
\end{example}

\begin{solution}
Put $z=e^{i\theta}$so that $\sin\theta=\frac{1}{2i}\left(z-\frac{1}{z}\right)$and
$d\theta=\frac{dz}{iz}$. Then 

\begin{align*}
\int_{0}^{2\pi}\frac{1}{5-4\sin\theta}d\theta & =\oint_{c}\frac{1}{5-4\left[\frac{1}{2i}\left(z-\frac{1}{z}\right)\right]}\frac{dz}{iz}\\
 & =\oint_{c}\frac{1}{5iz-2z^{2}+2}dz\end{align*}
Poles of integrand are given by 

\[
-2z^{2}+5iz+2=0\]
 or \[
z=\frac{-5i\pm\sqrt{-25+16}}{-4}=\frac{-5\pm3i}{-4}=2i,\frac{i}{2}\]
Only $z=\frac{i}{2}$ lies inside $c$. Residue at the simple pole
at $z=\frac{i}{2}$ is 

\[
\lim_{z\rightarrow\frac{i}{2}}\left(z-\frac{i}{2}\right)\times\left(\frac{1}{(2z-i)(-z+2i)}\right)=\frac{1}{3i}\]
Hence by Cauchy's residue theorem 

\[
\int_{0}^{2\pi}\frac{1}{5-4\sin\theta}d\theta=2\pi i\times Res\]


\[
=2\pi\times\frac{1}{3i}=\frac{2\pi}{3}\]
\end{solution}

\begin{example}
Using complex variables, evaluate the real integral
\[\ds \int_{0}^{2\pi}\frac{d\theta}{1-2p\sin\theta+p^{2}}\] where $p^{2}<1$
\end{example}
\begin{solution}

We have,

\[
I=\int_{0}^{2\pi}\frac{d\theta}{1-2p\sin\theta+p^{2}}=\int_{0}^{2\pi}\frac{d\theta}{1-2p\frac{(e^{i\theta}-e^{-i\theta})}{2i}+p^{2}}\]
Put $z=e^{i\theta},dz=ie^{i\theta}d\theta=\frac{dz}{zi}$

\begin{align*}
I & =\int_{c}\frac{1}{1+ip(z-\frac{1}{z})+p^{2}}\frac{dz}{zi}\\
 & =\int_{c}\frac{dz}{zi-pz^{2}+p+p^{2}zi} =\int_{c}\frac{dz}{(iz+p)(izp+1)} 
\end{align*}
Pole are given by 
\[(iz+p)(ipz+1)=0\]
\[z=-\frac{p}{i},-\frac{1}{ip} \text{ or } z=ip,\frac{i}{p}\]
$ip$ is the only poles inside the unit circle. Residue at$(z=ip)$
\[=\lim_{z\rightarrow ip}\frac{(z-pi)}{(iz+p)(izp+1)}=\lim_{z\rightarrow ip}\frac{1}{i(izp+1)}=\frac{1}{i}\left(\frac{1}{-p^{2}+1}\right)\]
Hence by Cauchy's residue theorem 
\[\int_{0}^{2\pi}\frac{d\theta}{1-2psin+p^{2}}=2\pi i\times Res=2\pi i\times\frac{1}{i}\left(\frac{1}{1-p^{2}}\right)=\frac{2\pi}{1-p^{2}}\]
\end{solution}
\subsection{Form II}
Form $\int_0^{2\pi} {\cos m\theta \over a+b \cos\theta} d\theta$, $\int_0^{2\pi} {\cos m\theta \over a+b \sin\theta} d\theta$, $\int_0^{2\pi} {\sin m\theta \over a+b \cos\theta} d\theta$ or $\int_0^{2\pi} {\sin m\theta \over a+b \sin\theta} d\theta$

\begin{example}
Using complex variable techniques evaluate the real integral $\int_{0}^{2\pi}\frac{\sin^{2}\theta d\theta}{5-4\cos\theta}$
\end{example}
\begin{solution}
We may write,\[
\int_{0}^{2\pi}\frac{\sin^{2}\theta d\theta}{5-4\cos\theta}=\frac{1}{2}\int_{0}^{2\pi}\frac{1-\cos2\theta}{5-4\cos\theta}d\theta\]
Put $z=e^{i\theta}$so that $\sin\theta=\frac{1}{2i}\left(z-\frac{1}{z}\right)$and
$d\theta=\frac{dz}{iz}$. Then 

\begin{align*}
\frac{1}{2}\int_{0}^{2\pi}\frac{1-\cos2\theta}{5-4\cos\theta}d\theta & =\text{Real part of }\frac{1}{2}\int_{0}^{2\pi}\frac{1-e^{2i\theta}}{5-4cos\theta}d\theta\\
 & =\text{Real part of }\frac{1}{2}\oint_{c}\frac{1-z^{2}}{5-2(z+\frac{1}{z})}(\frac{dz}{iz})\\
 & =\text{Real part of }\frac{1}{2i}\oint_{c}\frac{z^{2}-1}{2z^{2}-5z+2}dz\end{align*}
Poles of integrand are given by \[
2z^{2}-5z+2=0\]
 \[
(2z-1)(z-2)=0\]
\[
z=\frac{1}{2},2\]
So poles inside the contour $c$ there is a simple pole at $z=\frac{1}{2}$.
Residue at the simple pole $(z=\frac{1}{2})$ is 

\[
\lim_{z\rightarrow\frac{1}{2}}(z-\frac{1}{2})\frac{z^{2}-1}{(2z-1)(z-2)}\]


\[
=\lim_{z\rightarrow\frac{1}{2}}\frac{z^{2}-1}{2(z-2)}=\frac{\frac{1}{4}-1}{2(\frac{1}{2}-2)}=\frac{1}{4}\]


\[
\oint_{c}\frac{z^{2}-1}{2z^{2}-5z+2}dz=2\pi i\times Res=\frac{2\pi i}{4}\]


\[
\therefore\int_{0}^{2\pi}\frac{sin^{2}\theta}{5-4cos\theta}d\theta=\frac{1}{2i}\oint_{c}\frac{z^{2}-1}{2z^{2}-5z+2}dz=\frac{1}{2i}\times\frac{2\pi i}{4}=\frac{\pi}{4}
\]\\
\end{solution}

\subsection{Form III} 
$\int_0^{2\pi} {1 \over (a+b \cos\theta)^2} d\theta$ or $\int_0^{2\pi} {1 \over (a+b \sin\theta)^2} d\theta$

\begin{example}
Use the residue theorem to show that \[
\int_{0}^{2\pi}\frac{d\theta}{(a+bcos\theta)^{2}}=\frac{2\pi a}{\left(a^{2}-b^{2}\right){}^{3/2}}\]
\end{example}
\begin{solution}
Put $e^{i\theta}=z,$so that $e^{i\theta}(id\theta)=dz$or $izd\theta=dz=d\theta=\frac{dz}{iz}$

\[
\int_{0}^{2\pi }\frac{d\theta}{(a+bcos\theta)^{2}}=\int_{c}\frac{1}{[a+\frac{b}{2}(z+\frac{1}{z})]^{2}}\frac{dz}{iz}\]
where $c$ is the unit circle $z=1$

\[
=\int_{0}\frac{-4izdz}{(bz^{2}+2az+b)^{2}}=-\frac{4i}{b^{2}}\int_{c}\frac{zdz}{(z^{2}+\frac{2az}{b}+1)^{2}}\]
The pole are given by putting the denominator

\[
(z^{2}+\frac{2a}{b}z+1)^{2}=0\]


\[
(z-\alpha)^{2}(z-\beta)^{2}=0\]
where\[
[\alpha+\beta=-\frac{2a}{b}],\alpha\beta=1\]


\[
\alpha=\frac{-\frac{2a}{b}+\sqrt{\frac{4a^{2}}{b^{2}}-4}}{2}=\frac{-a+\sqrt{a^{2}-b^{2}}}{b}\]


\[
\beta=\frac{-\frac{2a}{b}-\sqrt{\frac{4a^{2}}{b^{2}}-4}}{2}=\frac{-a-\sqrt{a^{2}-b^{2}}}{b}\]


There are two poles at $z=\alpha$ and $z=\beta$, each of order 2.

Now,

Let\[
f(Z)=\frac{-4iz}{b^{2}(z-\alpha)^{2}(z-\beta)^{2}}=\frac{\phi(z)}{(z-\alpha)^{2}}\]
where $\ds \phi(z)=\frac{-4iz}{b^{2}(z-\beta)^{2}}$. Here $z=\alpha$ is
only point inside circle $c$. Residue at the double pole $z=\alpha$ 
\[=\lim_{z\rightarrow\alpha}\left(\frac{d}{dz}(z-\alpha)^{2}\frac{\phi(z)}{(z-\alpha)^{2}}\right)=\lim_{z\rightarrow\alpha}\frac{d}{dz}\phi(z)\]
\[=-\lim_{z\rightarrow\alpha}\frac{4i}{b^{2}}\left[\frac{(z-\beta)^{2}.1-z.2(z-\beta)}{(z-\beta)^{4}}\right]=-\frac{4i}{b^{2}}\lim_{z\rightarrow\alpha}\left[\frac{(z-\beta)-2z}{(z-\beta)^{3}}\right]\]
\[=\frac{4i}{b^{2}}\frac{\alpha+\beta}{(\alpha-\beta)^{3}}=\frac{4i}{b^{2}}\frac{-\frac{2a}{b}}{\left(\frac{2\sqrt{a^{2}-b^{2}}}{b}\right)^{3}}=\frac{-ai}{(a^{2}-b^{2})^{\frac{3}{2}}}\]
Hence \[\int_{0}^{2\pi}\frac{d\theta}{(a+bcos\theta)^{2}}=2\pi i\times\frac{-ai}{(a^{2}-b^{2})^{\frac{3}{2}}}=\frac{2\pi a}{(a^{2}-b^{2})^{\frac{3}{2}}}\]
\end{solution}
\subsection{Form IV}
$\int_0^{2\pi} {1 \over a^2 + \cos ^2 \theta} d\theta$, $\int_0^{2\pi} {1 \over a^2 + \sin^2 \theta}d\theta$
\begin{example}
Show by the method of residue, that

\[
\int_{0}^{\pi}\frac{ad\theta}{a^{2}+sin^{2}\theta}=\frac{\pi}{\sqrt{1+a^{2}}}\]

\end{example}
\[
I=\int_{0}^{\pi}\frac{ad\theta}{a^{2}+sin^{2}\theta}=\int_{0}^{\pi}\frac{2ad\theta}{2a^{2}+2sin^{2}\theta}\]


\[
\because(cos2\theta=1-2sin^{2})\]


\[
=\int_{0}^{\pi}\frac{2ad\theta}{2a^{2}+1-cos2\theta}=\int_{0}^{2\pi}\frac{ad\phi}{2a^{2}+1-cos\phi}\]
Putting $2\theta=\phi$

\[
=\int_{0}^{2\pi}\frac{ad\phi}{2a^{2}+1-\frac{1}{2}(e^{i\phi}+e^{-i\phi})}=\int_{0}^{2\pi}\frac{2ad\phi}{4a^{2}+2-(e^{i\phi}+e^{-i\phi})}\]


Now let $z=e^{i\phi}\;\Rightarrow d\phi\frac{dz}{iz}$

\begin{align*}
I & =\int_{0}^{2\pi}\frac{2a}{4a^{2}+2-(z+\frac{1}{z})}\frac{dz}{iz}\\
 & =2ai\int_{0}^{2\pi}\frac{dz}{\left[z^{2}-(4a^{2}+2)z-1\right]}\end{align*}
The pole are given by putting the denominator

\[
z^{2}-(4a^{2}+2)z-1=0\]


\[
z=\frac{(4a^{2}+2)\pm\sqrt[]{(4a^{2}+2)^{2}-4}}{2}=\frac{(4a^{2}+2)\pm\sqrt[]{16a^{4}+16a^{2}}}{2}\]


\[
=(2a^{2}+1)\pm2a\sqrt[]{a^{2}+1}\]


Let

\[
\alpha=(2a^{2}+1)+2a\sqrt[]{a^{2}+1}\]


\[
\beta=(2a^{2}+1)-2a\sqrt[]{a^{2}+1}\]


Let $z^{2}-(4a^{2}+2)z+1=(z-\alpha)(z-\beta)$. Therefore product
of the roots $=\alpha\beta=1$ or $|\alpha\beta|=1$. But $|\alpha|>1$
therefore $|\beta|<1$.

i.e., Only $\beta$ lies inside the circle $c$.

Residue (at $z=\beta$) is \begin{align*}
 & =\lim_{z\rightarrow\beta}(z-\beta)\frac{2ai}{(z-\alpha)(z-\beta)} =\frac{-2ai}{(\beta-\alpha)}\\
 & =\frac{2ai}{\left[(2a^{2}+1)-2a\sqrt{a^{2}+1}\right]-\left[(2a^{2}+1)+2a\sqrt{a^{2}+1}\right]}\\
 & =\frac{2ai}{-4a\sqrt{a^{2}+1}}=\frac{-i}{2\sqrt{a^{2}+1}}
\end{align*}
Hence by Cauchy's residue theorem
\[I=2\pi i\times Res\]
\[
=2\pi i\frac{-i}{2\sqrt[]{a^{2}+1}}=\frac{\pi}{\sqrt[]{a^{2}+1}}\]

\begin{problems}
\prob Use Residue of calculus to evaluate the following integrals:

	\subprob  $ \ds \int_0^{\pi}{1 \over 3+2\cos\theta} d\theta $
		\begin{sol}
	$\ds \frac{\pi}{2\sqrt{5}}$	
	\end{sol}
	\subprob  $ \ds \int_0^{2\pi}{\cos \theta \over 3+\sin\theta} d\theta $
	\begin{sol}
	0 ; \textbf{Hint}
	\[I = \int_0^{2\pi}{\cos \theta \over 3+\sin\theta} d\theta = Real part of \int_0^{2\pi}{\frac{e^{i\theta}}{3+\sin\theta}} d\theta \]
	\end{sol}
	\subprob  $ \ds \int_0^{\pi}{1+2\cos\theta \over 5+4\sin\theta} d\theta $
		\begin{sol}
	0 
	\end{sol}
	\subprob  $ \ds \int_0^{2\pi}{\sin^2 \theta - 2 \cos\theta \over 2+\cos\theta} d\theta $
			\begin{sol}
	$\frac{2\pi}{\sqrt{3}}$
	\end{sol}

	\subprob  $ \ds \int_0^{2\pi}{\cos 3\theta \over 5-4\cos\theta} d\theta $
	\begin{sol}
	$\frac{\pi}{12}$
	\end{sol}
	\subprob  $ \ds \int_0^{2\pi} e^{\cos \theta}[\cos(\sin \theta - n \theta)]d\theta $
	\begin{sol}
	Let 
	\begin{align*}
	I &=\int_0^{2\pi} e^{\cos \theta}[\cos(\sin \theta - n \theta+ i\sin(\sin \theta - n \theta)]d\theta \\
		&= \int_0^{2\pi}e^{e^{i\theta}}e^{-in\theta} 
\end{align*}
	Put $z=e^{i\theta}$,
	\[I = \int_C \frac{e^z}{iz^{n+1}}dz\]
	Now by residue theorem,
	\[I = \frac{2\pi}{n!}\]
	Now compare real parts to obtain
	\[\int_0^{2\pi} e^{\cos \theta}[\cos(\sin \theta - n \theta)]d\theta = \frac{2\pi}{n!}\]
	
	\end{sol}

	
%		\begin{sol}
%	$\ds \frac{2\pi}{\sqrt{5}}(3-\sqrt{5})^n,\;n>0$	
%	\end{sol}
	{\subprob  $ \ds \int_0^{2\pi}{(1+2\cos\theta)^n \cos n\theta \over 3+2\cos\theta} d\theta $}{	\subprob  $ \ds \int_0^{2\pi}{\cos 2\theta \over 1-2p\sin\theta + p^2} d\theta $ where $p^2 < 1$.}
					
\end{problems}

\subsection{Some Important Results}
\subsection*{Jordan's Inequality}\index{Jordan's Inequality} Consider the relation $y=\cos \theta$. As $\theta$ increases, $\cos\theta$ decreases and therefore $y$ decreases. The mean ordinate between $0$ and $\theta$ is $ \ds 
{1 \over \theta} \int_0^{\theta} \cos\theta d\theta = {\sin\theta \over \theta} $
which implies when 
\[0<\theta < {\pi \over 2} \text{ then } \frac{2}{\pi}< {\sin\theta \over \theta} < 1\]
\begin{theorem}
Let $AB$ be the arc $\alpha < \theta < \beta$ of the circle $|z-a|=r$. If ~~$\lim_{z \rightarrow a} (z-a) f(z) = k$ then
\[\lim_{r \rightarrow 0} \int_{AB} f(z) dz = i(\beta - \alpha)k\]
\end{theorem}
\begin{theorem}
Let $AB$ be the arc $\alpha < \theta < \beta$ of the circle $|z|=R$. If ~~$\lim_{z \rightarrow \infty} z.f(z) = k$ then
\[\lim_{R \rightarrow \infty} \int_{AB} f(z) dz = i(\beta - \alpha)k\]
\end{theorem}
\begin{theorem}
If $f(z) \rightarrow 0$ uniformly as $|z| \rightarrow \infty$, then  $\lim_{R \rightarrow \infty} \int_{C_R} e^{imz}f(z) dz = 0$, where $C_R$ denotes  the semicircle $|z|=R, ~~~m>0$. 
\end{theorem}
%\section{Evaluation of Real Integrals by Contour Integration}
\subsection{Type $\int_{-\infty}^{\infty} f(x) dy$}
Consider the integral $\int_{-\infty}^{\infty} f(x) dy$, such that $f(x) = {\phi (x) \over \psi (x)}$, where $\psi (x)$ has no real roots and the degree of $\psi (x)$ is greater than $\phi (x)$.
\paragraph{Procedure}
Consider a function $f(z)$ corresponding to function $f(x)$. 
Again consider the integral $\int_C f(z) dz$, where C is a curve, consisting of upper half of the circle $|z|=R$, and part of real axis from $-R$ to $R$. Here $R$ is on our choice and can be taken as such that there is no singularity  on its circumference $C_R$.

Now by Cauchy's theorem,we have
\[\int_C f(z) dz = 2\pi i . \sum R_k\]
\[\int_{-R}^R f(x) dx + \int_{C_R} f(z) dz = 2\pi i . \sum R_k\]
Now as,
\[~~~~~~~~~~~~ \lim_{R \rightarrow \infty}\int_{C_R} f(z) dz = 0 \]
We get, 
\[\int_{-\infty}^{\infty} f(x) dx = 2\pi i . \sum R_k\]
Which is required integral.														

\begin{example}
Evaluate $\int_{-\infty}^{\infty}\frac{x^{2}dx}{(x^{2}+1)(x^{2}+4)}$.
\end{example}
\begin{solution}
Consider the integral $\int_{C}f(z)dz$ where $f(z)=\frac{z^{2}}{(z^{2}+1)(z^{2}+4)}$
and $C$ is the contour consisting of the semi circle $C_{R}$ which
is upper half of a large circle $|z|=R$ of radius $R$ together with
the part of the real axis from$-R$ to $+R$. 

For the poles 
\[(z^{2}+1)(z^{2}+4)=0 \;\; \Rightarrow\; z=\pm i,z=\pm2i\]
So $z=i,2i$ are the only poles inside $C$. 

The residue at $z=i$
\begin{align*}
= & \lim_{z\rightarrow i}(z-i)\frac{z^{2}}{(z+i)(z-i)(z^{2}+4)}\\
= & \lim_{z\rightarrow i}\frac{z^{2}}{(z+i)(z^{2}+4)} = \frac{-1}{2i(-1+4)}=-\frac{1}{6i}
\end{align*}
The residue at $z=2i$
\begin{align*}
= & \lim_{z\rightarrow2i}(z-2i).\frac{z^{2}}{(z^{2}+1)(z+2i)(z-2i)}\\
= & \lim_{z\rightarrow2i}\frac{z^{2}}{(z^{2}+1)(z+2i)}=  \frac{(2i)^{2}}{(-4+1)(2i+2i)}=\frac{1}{3i}
\end{align*}
By theorem of residue;
\begin{align*}
\int_{C}f(z)dz= & 2\pi i[Res(i)+Res(2i)]\\
= & 2\pi i(-\frac{1}{6i}+\frac{1}{3i})=\frac{\pi}{3}\end{align*}


i.e. \[\int_{-R}^{R}f(x)dx+\int_{C_{R}}f(z)dz=\frac{\pi}{3}\]
Hence by making $R\rightarrow\infty,$ relation (1) becomes 
\[\int_{-\infty}^{\infty}f(x)dx+\lim_{R\rightarrow\infty}\int_{C_{R}}f(z)dz=\frac{\pi}{3}\]
Now
\begin{align*}
\left|\int_{C_{R}}f(z)dz\right| & =\left|\int_{C_{R}}\frac{z^{2}dz}{(z^{2}+1)(z^{2}+4)}\right|\\
 & \leq\int_{C_{R}}\frac{\left|z^{2}dz\right|}{\left|(z^{2}+1)\right|\left|(z^{2}+4)\right|}\\
 & \leq\int_{C_{R}}\frac{\left|z^{2}dz\right|}{\left|(|z|^{2}-1)\right|\left|(|z|^{2}-4)\right|}\\
 & \;\;\;\;\;\;\;\;\;\;\;\;\;\;\;\;\;\;\;\;\;\because|z|=R\\
 & \;\;\;\;\;\;\;\;\;\;\;\;\;\;\;\;\;\;\;\;\;\;\; z=Re^{i\theta},0<\theta<\pi\\
 & \;\;\;\;\;\;\;\;\;\;\;\;\;\;\;\;\;\;\;\;\;\;\;|dz|=Rd\theta,\\
 & \leq\int_{C_{R}}\frac{R^{2}Rd\theta}{(R^{2}-1)(R^{2}-4)}\\
 & =\frac{\pi R^{3}}{(R^{2}-1)(R^{2}-4)}\rightarrow0\text{ as }R\rightarrow\infty\end{align*}
Thus
\[\int_{-\infty}^{\infty}f(x)dx=\frac{\pi}{3}\]
i.e.,\[\int_{-\infty}^{\infty}\frac{x^{2}dx}{(x^{2}+1)(x^{2}+4)}=\frac{\pi}{3}\]
\end{solution}
\begin{example}
Evaluate by the model of complex variables , the integral $\ds \int_{-\infty}^{\infty}\frac{x^{2}}{(1+x^{2})^{3}}dx$
\end{example}
\begin{solution}
Consider the integral $\int_{C}f(z)dz$ where $f(z)=\frac{z^{2}}{(1+z^{2})^{3}}$
and $C$ is the contour consisting of the semi circle $C_{R}$ which
is upper half of a large circle $|z|=R$ of radius $R$ together with
the part of the real axis from$-R$ to $+R$. For the poles 
\[(1+z^{2})^{3}=0\]
\[\Rightarrow\; z=\pm i\]
$\therefore z=i$ and $z=-i$ are the two poles each of order 3. But
only $z=i$ lies within the $C$. Residue at $z=i$

To get residue at $z=i$ , put $z=i+t$ , then 
\begin{align*}
\frac{z^{2}}{(1+z^{2})^{3}} & =\frac{(i+t)^{2}}{[1+(i+t)^{2}]^{3}}=\frac{-1+2it+t^{2}}{[1-1+2it+t^{2}]^{3}}\\
 & =\frac{(-1+2it+t^{2})}{(2it)^{3}(1+\frac{1}{2i}t)^{3}}=\frac{(-1+2it+t^{2})}{-8it^{3}}(1+\frac{t}{2i})^{-3}\\
 & =-\frac{1}{8i}\left(-\frac{1}{t^{3}}+\frac{2i}{t^{2}}+\frac{1}{t}\right)\left(1-\frac{3t}{2i}+\frac{(-3)(-4)}{2}\frac{t^{2}}{-4}+...\right)\\
 & =-\frac{1}{8i}\left[-\frac{1}{t^{3}}+\frac{2i}{t^{2}}+\frac{1}{t}\right]\left[1-\frac{3}{2i}t-\frac{3}{2}t^{2}+...\right]
 \end{align*}
Here coefficient of $\ds \frac{1}{t}$ is $\ds \frac{-1}{8i}(\frac{3}{2}-3+1)$ = $\ds \frac{-i}{16}$, which is therefore the residue at $z=i$.

Using Cauchy's theorem of residues we have 
\[\int_{c}f(z)dz=2\pi i\times Res\]
where $Res$ = Sum of the residues of $f(z)$ at the poles within $c$.
\[\int_{-R}^{R}f(x)dx+\int_{C_{R}}f(z)dz=2\pi i(-\frac{i}{16})\]
\[\int_{-R}^{R}\frac{x^{2}}{(1+x^{2})^{3}}dx+\int_{C_{R}}\frac{z^{2}}{(1+z^{2})^{3}}dz=\frac{\pi}{8}\]
Now 
\begin{align*}
\left|\int_{C_{R}}\frac{z^{2}}{(1+z^{2})^{3}}dz\right| & \le\int_{C_{R}}\frac{\left|z\right|^{2}|dz|}{\left|1+z^{2}\right|{}^{3}}
  \le\int_{C_{R}}\frac{|z|^{2}|dz|}{\left(\left|z^{2}\right|-1\right)^{3}}\\
 & \;\;\;\;\;\;\;\;\;\;\;\;\;\text{Since }z=Re^{i\theta},|dz|=Rd\theta\\
 & \le\frac{R^{2}}{(R^{2}-1)^{3}}\int_{0}^{\pi}Rd\theta\\
 & =\frac{\pi R^{3}}{(R^{2}-1)^{3}}\rightarrow0\text{ as }R\rightarrow\infty\end{align*}
Hence
\[\int_{-\infty}^{\infty}\frac{x^{2}}{(1+x^{2})^{3}}dx=\frac{\pi}{8}\]
\end{solution}
\begin{example}
Using the complex variables techniques, evaluate the integral
\[\int_{0}^{\infty}\frac{dx}{x^{4}+16}\]
\end{example}
\begin{solution}
Consider the integral $\int_{C}f(z)dz$ where $f(z)=\frac{1}{z^{4}+16}$
and $C$ is the contour consisting of the semi circle $C_{R}$ which
is upper half of a large circle $|z|=R$ of radius $R$ together with
the part of the real axis from$-R$ to $+R$. 

For the poles 
\[z^{4}+16=  0 \; \Rightarrow \; z^{4}=  -16 \; \Rightarrow \; \Rightarrow \; z^{4}=  16e^{i\pi}=16e^{i(2n+1)\pi} \;\Rightarrow \;
z=  2e^{(2n+1)i\pi/4} \]
Now for $n=0,1,2,3
$\begin{align*}
z_{1}=2e^{i\pi/4} & 2(cos\frac{\pi}{4}+isin\frac{\pi}{4})=2(\frac{1}{\sqrt{2}}+i\frac{1}{\sqrt{2}})=\sqrt{2}(1+i)\\
z_{2}=2e^{3i\pi/4} & 2(cos\frac{3\pi}{4}+isin\frac{3\pi}{4})=2(-\frac{1}{\sqrt{2}}+i\frac{1}{\sqrt{2}})=\sqrt{2}(-1+i)\\
z_{3}=2e^{5i\pi/4} & 2(cos\frac{5\pi}{4}+isin\frac{5\pi}{4})=2(-\frac{1}{\sqrt{2}}-i\frac{1}{\sqrt{2}})=-\sqrt{2}(1+i)\\
z_{4}=2e^{7i\pi/4} & 2(cos\frac{7\pi}{4}+isin\frac{7\pi}{4})=2(\frac{1}{\sqrt{2}}-i\frac{1}{\sqrt{2}})=\sqrt{2}(1-i)
\end{align*}
There are four poles , but only two poles at $z_{1}$ and $z_{2}$
lie within $C$. 

Residue (at $z=2e^{\frac{i\pi}{4}}$)\[
=\left[\frac{1}{\frac{d}{dz}(z^{4}+16)}\right]_{z=2e^{\frac{i\pi}{4}}}=\left[\frac{1}{4z^{3}}\right]_{z=2e^{\frac{i\pi}{4}}}=\frac{1}{4(2e^{i\frac{\pi}{4}})^{3}}=\frac{1}{32e^{i\frac{3\pi}{4}}}=-\frac{e^{i\frac{\pi}{4}}}{32}\]


Residues (at $z=2e^{\frac{3i\pi}{4}}$)\[
=\left[\frac{1}{\frac{d}{dz}(z^{4}+16)}\right]_{z=2e^{\frac{3i\pi}{4}}}=\left[\frac{1}{4z^{3}}\right]_{z=2e^{\frac{3i\pi}{4}}}=\frac{1}{4(2e^{3i\frac{\pi}{4}})^{3}}=\frac{1}{32e^{i\frac{9\pi}{4}}}=\frac{e^{-i\frac{\pi}{4}}}{32}\text{ Note Here.}\]
\[\int_{c}f(z)dz=2\pi i\times\left(\frac{-e^{i\pi/4}+e^{-i\pi/4}}{32}\right)\]
\[\int_{c}f(z)dz=-\left(2\pi i\frac{i\sin\frac{\pi}{4}}{16}\right)=\frac{\sqrt{2}\pi}{16}\]
where $Res$= Sum of residues at poles within $C$
\[\int_{-R}^{R}f(z)dz+\int_{C_{R}}f(z)dz==\frac{\sqrt{2}\pi}{16}\]
\[\int_{-R}^{R}\frac{1}{x^{4}+16}dx+\int_{C_{R}}\frac{1}{z^{4}+16}dz=\frac{\sqrt{2}\pi}{16}\]
 Now \begin{align*}
\left|\int_{C_{R}}\frac{1}{z^{4}+16}dz\right| & \le\int_{C_{R}}\frac{|dz|}{\left|z^{4}+16\right|}\\
 & \le\int_{C_{R}}\frac{|dz|}{(\left|z\right|^{4}-16)}\\
 & \;\;\;\;\;\;\;\;\;\;\;\;\;\text{Since }z=Re^{i\theta},|dz|=Rd\theta\\
 & \le\int_{0}^{\pi}\frac{Rd\theta}{R^{4}-16}=\frac{R\pi}{R^{4}-16}\rightarrow0\text{ as }R\rightarrow\infty\end{align*}
Hence \[\int_{-\infty}^{\infty}\frac{1}{x^{4}+16}dx==\frac{\sqrt{2}\pi}{16}\]
\[2\int_{0}^{\infty}\frac{1}{x^{4}+16}dx=\frac{\sqrt{2}\pi}{16}\]
\[\int_{0}^{\infty}\frac{1}{x^{4}+16}dx=\frac{\sqrt{2}\pi}{32}\]
\end{solution}

\begin{example}
Using complex variables, evaluate the real integral

\[
\int_{0}^{\infty}\frac{cos(3x)dx}{(x^{2}+1)(x^{2}+4)}\]

\end{example}
\begin{solution}
Consider the integral $\int_{C}f(z)dz$ where $f(z)=\frac{e^{3iz}}{(z^{2}+1)(z^{2}+4)}$
and $C$ is the contour consisting of the semi circle $C_{R}$ which
is upper half of a large circle $|z|=R$ of radius $R$ together with
the part of the real axis from$-R$ to $+R$. 

For the poles 
\begin{center}
$(z^{2}+1)(z^{2}+4)=0$ $\Rightarrow z^{2}+1=0$ or $z=\pm i$  $\Rightarrow z^{2}+4=0$ or $z=\pm2i$
\end{center}
The Poles at $z=i$ and $z=2i$ lie within the contour.

Residue (at $z=i$)\[=\lim_{z\rightarrow i}\frac{(z-i)e^{3iz}}{(z^{2}+1)(z^{2}+4)}=\lim_{z\rightarrow i}\frac{e^{3iz}}{(z+i)(z^{2}+4)}=\frac{e^{-3}}{6i}\]
Residue (at $z=2i$)\[=\lim_{z\rightarrow2i}\frac{(z-2i)e^{3iz}}{(z^{2}+1)(z^{2}+4)}=\lim_{z\rightarrow i}\frac{e^{3iz}}{(z^{2}+1)(z+2i)}=-\frac{e^{-6}}{12i}\]
By theorem of Residue \[\int_{C}f(z)dz=2\pi i\times Res\]
\[\int_{-R}^{R}\frac{e^{3iz}dz}{(z^{2}+1)(z^{2}+4)}+\int_{C_{R}}\frac{e^{3iz}dz}{(z^{2}+1)(z^{2}+4)}=2\pi i\left[\frac{e^{-3}}{6i}+\frac{e^{-6}}{-12i}\right]\]
\[\lim_{R\rightarrow\infty}\int\frac{e^{3izdz}}{(z^{2}+1)(z^{2}+4)}=0,\text{ By Jordan's Lemma}\]
\[\int_{-\infty}^{\infty}\frac{e^{3ix}}{(x^{2}+1)(x^{2}+4)}dx=\pi\left[\frac{e^{-3}}{3}-\frac{e^{-6}}{6}\right]\]
\[\int_{-\infty}^{\infty}\frac{cos(3x)dx}{(x^{2}+1)(x^{2}+4)}=\text{ Real part of}\int_{-\infty}^{\infty}\frac{e^{3ix}dx}{(x^{2}+1)(x^{2}+4)}\]
\[\int_{-\infty}^{\infty}\frac{cos(3x)dx}{(x^{2}+1)(x^{2}+4)}=\pi\left[\frac{e^{-3}}{3}-\frac{e^{-6}}{6}\right]\]
\[\int_{0}^{\infty}\frac{cos(3x)dx}{(x^{2}+1)(x^{2}+4)}=\frac{\pi}{2}\left[\frac{e^{-3}}{3}-\frac{e^{-6}}{6}\right]\]
\end{solution}

\begin{problems}
\prob Use Residue of calculus to evaluate the following integrals:
\subprob  $ \ds \int_0^{\infty} {\cos mx \over (x^2+1)} dx $
\begin{sol}
$\frac{\pi e^{-m}}{2}$
\end{sol}
\subprob  $ \ds \int_0^{\infty} {\log (1+x^2) \over (x^2+1)} dx $
\begin{sol}
$\pi \log 2$
\end{sol}
\subprob  $ \ds \int_0^{\infty} {1 \over (x^2+1)^3} dx $
\begin{sol}
$\frac{3\pi}{16}$
\end{sol}
\subprob  $ \ds \int_0^{\infty} {1 \over (x^6+1)} dx $
\begin{sol}
$\frac{\pi}{3}$
\end{sol}
\subprob  $ \ds \int_0^{\infty} {\cos x^2 + \sin x^2 -1 \over (x^4+16)} dx $
\prob  By contour integration, prove that $ \ds \int_0^{\infty} {\sin mx \over x} dx = {\pi \over 2} $
\prob  Show that, if $a \geq b \geq 0$, then 	$ \ds \int_0^{\infty} {\cos 2ax -\cos 2bx \over x^2} dx = \pi (b-a) $
\prob  Show that $ \ds \int_0^{\infty} {x^3 \sin x  \over (x^2+a^2)(x^2+b^2)} dx = \frac{\pi}{2(a^2-b^2)}[a^2e^{-a}-b^2e^{-b}]$ where $a>b>0$	
\end{problems}

