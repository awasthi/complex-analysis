\chapter{Complex Integration}
\section{Integration in complex plain}
In case of real variable, the path of the integration of $\ds \int_a^b f(x)dx$ is always along the $x$-axis from $x=a$ to $x=b$. But in case of a complex function $f(z)$ the path of a complex function $f(z)$ the path of the definite integral $\ds \int_{\alpha}^{\beta} f(z)dz$ can be along any curve from $z=\alpha$ to $z=\beta$. Here we will consider some examples.
\begin{example}
Evaluate $\ds \int_{0}^{2+i} \bar{z}^2 dz$ along the real axis from $z=0$ to $z=2$ and then along parallel to $y$-axis from $z=2$ to $z=2+i$.
\end{example}
\begin{solution}
\begin{align*}
	\int_{0}^{2+i} \bar{z}^2 dz		&= \int_{0}^{2+i} (x-iy)^2 (dx+idy)\\
																&= \int_{0}^{2+i} (x^2-y^2-2ixy) (dx+idy)\\
\end{align*}
\paragraph{Along real axis from $z=0$ to $z=2$ (y=0)}
\[y=0 \Rightarrow dy=0, dz = d(x+iy) = dx\]
\[z=0, y=0 \Rightarrow x=0~~~~~and~~~~~z=2, y=0 \Rightarrow x=2\]
\begin{align*}
	\int_{0}^{2+i} \bar{z}^2 dz		&= \int_{0}^{2} (x^2) (dx)\\
	&= \left[\frac{x^3}{3} \right]_0^2 = \frac{8}{3}\\
\end{align*}
\paragraph{Along parallel to $y$-axis from $z=2$ to $z=2+i$ (x=2)}
\[x=2 \Rightarrow dx=0, dz = d(x+iy) = idy\]
\[z=2, x=2 \Rightarrow y=0~~~~~and~~~~~z=2+i, x=2 \Rightarrow y=1\]
\begin{align*}
	\int_{0}^{2+i} \bar{z}^2 dz		&= \int_{0}^{1} (4-y^2-4iy) (i.dy)\\
	&= i \left[4y - \frac{y^3}{3} - 4i\frac{y^2}{2} \right]_0^1 =i\left[\frac{11}{3}i + 2\right]\\
\end{align*}
$\ds \int_{0}^{2+i} \bar{z}^2 dz$ along the real axis from $z=0$ to $z=2$ then along parallel to $y$-axis from $z=2$ to $z=2+i$
\[= \frac{8}{3} +  \frac{11}{3}i + 2  = \frac{1}{3}(14+11i)\]
\end{solution}
\begin{problems}
\prob Find the value of the integral 
\[\int_0^{1+i}(x-y+ix^2)dz\]
\subprob Along the straight line from $z=0$ to $z=1+i$.
\subprob along the real axis from $z=0$ to $z=1$ and then along parallel to $y$-axis from $z=1$ to $z=1+i$.
\prob Integrate $f(z) = x^2 + ixy$ from $A(1,1)$ to $B(2,8)$ along
\subprob the straight line $AB$
\subprob the curve $C$, $x=t, ~y=t^3$.
\prob Evaluate the integral $\ds int_c (3y^2dx+2ydy)$, where $c$ is the circle $x^2+y^2=1$, counterclockwise from $(1,0)$ to $(0,1)$.

\end{problems}

\section{Cauchy's Integral Theorem}
\begin{thm}
If a function $f(z)$ is analytic and its derivative $f'(z)$ continuous at all points within and on a simple closed curve $c$, then $\ds \int_c f(z) dz = 0$.
\end{thm}
\begin{proof}
Let $f(z)=u+iv$ and $z=x+iy$ and region enclosed by the curve $c$ be $R$, then
\begin{align*}
	\int_cf(z)dz 	&= \int_c(u+iv)(dx+idy) =  \int_c(udx-vdy) + i(vdx+udy)\\
								&= \int\int_R (-v_x - u_y)dxdy + i\int\int_R (u_x - v_y)dxdy\\
								\text{By Cauchy-Riemann equations,} \\
								&= \int\int_R (u_y - u_y)dxdy + i\int\int_R (u_x - u_x)dxdy =0
\end{align*}
\end{proof}
\begin{example}
Find the integral $\ds \int_c \frac{3z^2+7z+1}{z+1}dz$, where $c$ is the circle $\ds |z|=\frac{1}{2}$.
\end{example}
\begin{solution}
Poles of integrad are given by
$$z+1 = 0$$
That is, $z=-1$. Since given circle $|z|=\frac{1}{2}$, with centre $z=0$ and radius $1/2$ does not enclose any singularity of the given function. Thus it is obvious that the integrand is analytic everywhere. Hence, by Cauchy's Theorem,
\[\int_c \frac{3z^2+7z+1}{z+1}dz = 0\]
\end{solution}
\begin{thm}[Cauchy's integral theorem for multiconnected region]
If a function $f(z)$ is analytic in region $R$ between two simple closed curves $c_1$ and $c_2$, then
\[\int_{c_1} f(z) dz = \int_{c_2} f(z) dz\]
\end{thm}
\begin{proof}
Since $f(z)$ is analytic in region $R$, By Cauchy's Theorem
\[\int f(z) dz = 0\]
where path of integration is along $AB$, and curves $C_2$ in clockwise direction and along $BA$ and along $C_1$ in anticlockwise direction.

We may write,
\[\int_{AB} f(z) dz - \int_{c_2} f(z) dz + \int_{BA} f(z) dz + \int_{c_1} f(z) dz = 0\]
or
\[- \int_{c_2} f(z) dz +  \int_{c_1} f(z) dz = 0\]
\[ \int_{c_1} f(z) dz =  \int_{c_2} f(z) dz \]
\end{proof}
\section{Cauchy Integral Formula}
\begin{thm}
If a function $f(z)$ is analytic within and on a closed curve $c$, and if $a$ is any point within $c$ , then
\[f(a) = \frac{1}{2\pi}\int_{z} \frac{f(z)}{(z-a)} dz\]
\end{thm}
\begin{proof}
Let $z=a$ be a point within a closed curve $c$. Describe a circle $\gamma$ such that $|z-a|=\rho$ and it lies entirely within $c$. Now consider the function 
\[\phi(z) = \frac{f(z)}{(z-a)}\]
Obviously, this function is analytic in region between $\gamma$ and $c$. Hence by Cauchy's integral theorem for multiconnected region, we have
\[\int_{c} \phi(z) dz = \int_{\gamma} \phi(z) dz\]
or
\begin{align*}
	\int_{c} \frac{f(z)}{(z-a)} dz 	&= \int_{\gamma} \frac{f(z)}{(z-a)} dz \\
																	&= \int_{\gamma} \frac{f(z) - f(a) + f(a)}{(z-a)} dz \\
																	&= \int_{\gamma} \frac{f(z) - f(a)}{(z-a)} dz + \int_{\gamma} \frac{f(a)}{(z-a)} dz \\
																	&= I_1 + I_2
\end{align*}
Now, since $|z-a|=\rho$, we have $z=a+ \rho e^{i\theta}$ and $dz = i \rho e^{i\theta} d\theta$. Hence
\begin{align*}
 I_1 &= \int_{\gamma} \frac{f(z) - f(a)}{(z-a)} dz \\
 	 &= \int_{0}^{2\pi} \frac{f(a+ \rho e^{i\theta}) - f(a)}{[(a+ \rho e^{i\theta})-a]} i \rho e^{i\theta} d\theta \\
 	 &= \int_{0}^{2\pi} [f(a+ \rho e^{i\theta}) - f(a)] i d\theta \\
 	 &=0 ~~~~~~~~\text{ as } \rho \text{ tends to 0}
\end{align*}
and
\begin{align*}
 I_2 &= \int_{\gamma} \frac{f(a)}{(z-a)} dz \\
 	 &= \int_{0}^{2\pi} \frac{f(a)}{[(a+ \rho e^{i\theta})-a]} i \rho e^{i\theta} d\theta \\
 	 &= f(a) \int_{0}^{2\pi} i d\theta \\
 	 &= 2\pi i f(a)	 
 \end{align*}
Hence,
\[\int_{c} \frac{f(z)}{(z-a)} dz = I_1 + I_2\]
That is 
\[\int_{c} \frac{f(z)}{(z-a)} dz = 0 + 2 \pi i f(a) \]
or
\[ f(a) = \frac{1}{2 \pi i}\int_{c} \frac{f(z)}{(z-a)} dz \]
\end{proof}
\begin{example}
Evaluate (i) $\ds \int_c \frac{e^z}{z+2}dz$ and (ii) $\ds \int_c \frac{e^z}{z}dz$, where $c$ is circle $|z|=1$.
\end{example}
\begin{solution}
(i) The function $\ds \frac{e^z}{z+2}$ is analytic everywhere except at $z=-2$. This point lies outside the circle $|z|=1$. Thus function is analytic within and on $c$, by Cauchy's Theorem, we have
\[\int_{|z|=1} \frac{e^z}{z+2}dz = 0\]

(ii)
The function $\ds \frac{e^z}{z}$ is analytic everywhere except at $z=0$. The point $z=0$ strictly inside $|z|=1$. Hence by Cauchy's Integral formula, we have
\[\int_{|z|=1}~ \frac{e^z}{z}dz = 2 \pi i (e^z)_{z=0} =  2 \pi i \]
\end{solution}
\section{Cauchy Integral Formula For The Derivatives of An Analytic Function}
\begin{thm}
If a function $f(z)$ is analytic within and on a closed curve $c$, and if $a$ is any point within $c$ , then its derivative is also analytic within and on closed curve $c$, and is given as 
\[f'(a) = \frac{1}{2\pi}\int_{c} \frac{f(z)}{(z-a)^2} dz\]
\end{thm}
\begin{proof}
We know Cauchy's Integral formula
\begin{align*}
	f(a) &= \frac{1}{2\pi}\int_{c} \frac{f(z)}{(z-a)} dz \\
\text{Differentiating, wrt $a$ , we get} \\
	f'(a) &= \frac{1}{2\pi}\frac{d}{da}\left[\int_{c} \frac{f(z)}{(z-a)} dz \right]\\
	      &= \frac{1}{2\pi} \int_{c} f(z) \frac{\partial}{\partial a}\left[\frac{1}{(z-a)} \right]dz\\
	      &= \frac{1}{2\pi}\int_{c} \frac{f(z)}{(z-a)^2} dz
\end{align*}
We may generalize it,
\[f^n(a) = \frac{n!}{2\pi}\int_{c} \frac{f(z)}{(z-a)^{n+1}} dz\]
\end{proof}
