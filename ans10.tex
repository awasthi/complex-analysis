\begin{Solution}{10.1.a}
	$\ds \frac{\pi}{2\sqrt{5}}$
	
\end{Solution}
\begin{Solution}{10.1.b}
	0 ; \textbf{Hint}
	\[I = \int_0^{2\pi}{\cos \theta \over 3+\sin\theta} d\theta = Real part of \int_0^{2\pi}{\frac{e^{i\theta}}{3+\sin\theta}} d\theta \]
	
\end{Solution}
\begin{Solution}{10.1.c}
	0
	
\end{Solution}
\begin{Solution}{10.1.d}
	$\frac{2\pi}{\sqrt{3}}$
	
\end{Solution}
\begin{Solution}{10.1.e}
	$\frac{\pi}{12}$
	
\end{Solution}
\begin{Solution}{10.1.f}
	Let
	\begin{align*}
	I &=\int_0^{2\pi} e^{\cos \theta}[\cos(\sin \theta - n \theta+ i\sin(\sin \theta - n \theta)]d\theta \\
		&= \int_0^{2\pi}e^{e^{i\theta}}e^{-in\theta}
\end{align*}
	Put $z=e^{i\theta}$,
	\[I = \int_C \frac{e^z}{iz^{n+1}}dz\]
	Now by residue theorem,
	\[I = \frac{2\pi}{n!}\]
	Now compare real parts to obtain
	\[\int_0^{2\pi} e^{\cos \theta}[\cos(\sin \theta - n \theta)]d\theta = \frac{2\pi}{n!}\]

	
\end{Solution}
\begin{Solution}{10.1.a}
$\frac{\pi e^{-m}}{2}$
\end{Solution}
\begin{Solution}{10.1.b}
$\pi \log 2$
\end{Solution}
\begin{Solution}{10.1.c}
$\frac{3\pi}{16}$
\end{Solution}
\begin{Solution}{10.1.d}
$\frac{\pi}{3}$
\end{Solution}
